\documentclass[]{article}
\usepackage{lmodern}
\usepackage{amssymb,amsmath}
\usepackage{ifxetex,ifluatex}
\usepackage{fixltx2e} % provides \textsubscript
\ifnum 0\ifxetex 1\fi\ifluatex 1\fi=0 % if pdftex
  \usepackage[T1]{fontenc}
  \usepackage[utf8]{inputenc}
\else % if luatex or xelatex
  \ifxetex
    \usepackage{mathspec}
  \else
    \usepackage{fontspec}
  \fi
  \defaultfontfeatures{Ligatures=TeX,Scale=MatchLowercase}
\fi
% use upquote if available, for straight quotes in verbatim environments
\IfFileExists{upquote.sty}{\usepackage{upquote}}{}
% use microtype if available
\IfFileExists{microtype.sty}{%
\usepackage{microtype}
\UseMicrotypeSet[protrusion]{basicmath} % disable protrusion for tt fonts
}{}
\usepackage[margin=1in]{geometry}
\usepackage{hyperref}
\hypersetup{unicode=true,
            pdftitle={9\_17\_1},
            pdfauthor={JayLim814},
            pdfborder={0 0 0},
            breaklinks=true}
\urlstyle{same}  % don't use monospace font for urls
\usepackage{graphicx,grffile}
\makeatletter
\def\maxwidth{\ifdim\Gin@nat@width>\linewidth\linewidth\else\Gin@nat@width\fi}
\def\maxheight{\ifdim\Gin@nat@height>\textheight\textheight\else\Gin@nat@height\fi}
\makeatother
% Scale images if necessary, so that they will not overflow the page
% margins by default, and it is still possible to overwrite the defaults
% using explicit options in \includegraphics[width, height, ...]{}
\setkeys{Gin}{width=\maxwidth,height=\maxheight,keepaspectratio}
\IfFileExists{parskip.sty}{%
\usepackage{parskip}
}{% else
\setlength{\parindent}{0pt}
\setlength{\parskip}{6pt plus 2pt minus 1pt}
}
\setlength{\emergencystretch}{3em}  % prevent overfull lines
\providecommand{\tightlist}{%
  \setlength{\itemsep}{0pt}\setlength{\parskip}{0pt}}
\setcounter{secnumdepth}{0}
% Redefines (sub)paragraphs to behave more like sections
\ifx\paragraph\undefined\else
\let\oldparagraph\paragraph
\renewcommand{\paragraph}[1]{\oldparagraph{#1}\mbox{}}
\fi
\ifx\subparagraph\undefined\else
\let\oldsubparagraph\subparagraph
\renewcommand{\subparagraph}[1]{\oldsubparagraph{#1}\mbox{}}
\fi

%%% Use protect on footnotes to avoid problems with footnotes in titles
\let\rmarkdownfootnote\footnote%
\def\footnote{\protect\rmarkdownfootnote}

%%% Change title format to be more compact
\usepackage{titling}

% Create subtitle command for use in maketitle
\newcommand{\subtitle}[1]{
  \posttitle{
    \begin{center}\large#1\end{center}
    }
}

\setlength{\droptitle}{-2em}

  \title{9\_17\_1}
    \pretitle{\vspace{\droptitle}\centering\huge}
  \posttitle{\par}
    \author{JayLim814}
    \preauthor{\centering\large\emph}
  \postauthor{\par}
      \predate{\centering\large\emph}
  \postdate{\par}
    \date{2018년 9월 19일}


\begin{document}
\maketitle

\section{9월 17일 수업.}\label{-17-.}

\section{Data Viualization 마무리. \& Graphics
공부}\label{data-viualization-.-graphics-}

\section{Statistical
transformations.}\label{statistical-transformations.}

library(tidyverse) \# 이제부터는 다이아몬드. 원래 데이터에서 없는 변수를
만들어내는 방법. \# 막대그래프 =\textgreater{} count는 원래 자료에 있지
않은 변수. 새로운 변수를 만들어 내는 과정. \# geom\_bar 대신에
stat\_count 오브젝트를 사용해도 괜찮음. ggplot(data = diamonds) +
geom\_bar(mapping = aes(x = cut)) \# 상대도수로도 바꿀 수 있음
ggplot(data = diamonds) + geom\_bar(mapping = aes(x = cut, y = ..prop..,
, group = 1)) ggplot(data = diamonds) + stat\_summary( \# 요약
그리기(자기 멋대로 가능) mapping = aes(x = cut, y = depth), \# y축은
depth fun.ymin = min, \# 미니멈은 min값 fun.ymax = max, \# 위쪽은 max 값
fun.y = median \# 점찍는 곳은 median 값.(mean으로도 수정가능) )

\section{Position Adjustment =\textgreater{}
위치조정}\label{position-adjustment-}

ggplot(data = diamonds) + geom\_bar(mapping = aes(x = cut, colour =
cut))

\section{색을 컷에 따라서 조정. 사실 우리가 원한는 걸로 하려면 fill =
cut이어야함}\label{---.------fill-cut}

ggplot(data = diamonds) + geom\_bar(mapping = aes(x = cut, fill = cut))

\section{이게 우리가 원하는 그림.}\label{---.}

\section{그럼 Fill 을 다른 것으로도 할 수 있지
않을까?}\label{-fill-------}

ggplot(data = diamonds) + geom\_bar(mapping = aes(x = cut, fill =
clarity)) \# 한 bar 안에서 색깔별로 clarity가 표시됨.

ggplot(data = diamonds) + geom\_bar(mapping = aes(x = cut, fill =
clarity), position = ``identity'') \# 를 하면 stacking을 할때, 0부터
무조건 쌓게됨. 그래서 안보이는 범주도 있음.

\section{dodge(피구) =\textgreater{} 서로 피해서 쌓으라}\label{dodge---}

ggplot(data = diamonds) + geom\_bar(mapping = aes(x = cut, fill =
clarity), position = ``dodge'')

\section{fill을 하게 되면 전부 1.0에 맞춰서
보여줌.}\label{fill----1.0--.}

ggplot(data = diamonds) + geom\_bar(mapping = aes(x = cut, fill =
clarity), position = ``fill'')

\section{stack은 차차 쌓아 나가라. 첨에 우리가 fill = clarity만
써놨던거랑 비슷하게 됨}\label{stack---.---fill-clarity---}

ggplot(data = diamonds) + geom\_bar(mapping = aes(x = cut, fill =
clarity), position = ``stack'')

\section{jitter는 dodge처럼 피하되, 랜덤하게 마구 피해라. 산점도에서
x값에 같은 애들에 약간씩 noise를
둠.}\label{jitter-dodge----.--x----noise-.}

\section{x축이 범주형 자료일 경우, x축이 같은것에 큰 의미가 없기 때문에,
간격을 채우기 위해서 사용하는 경우가 있음}\label{x----x-----------}

\subsection{Coordinate system : 좌표계}\label{coordinate-system-}

ggplot(data = mpg, mapping = aes(x = class, y = hwy)) + geom\_boxplot()

\section{상자 그림 그리는 법. 각 class 별 hwy를 boxplot으로 구분 있음.
가령 subcompact의(준 소형차) 연비가 좋다는 것을 확인
가능.}\label{---.--class--hwy-boxplot--.--subcompact------.}

\section{suv의 경우 연비가 낮음. pickup은 트럭임. 이것도 연비 낮음.
이거는 좌표계가 기본적으로 cartesnion(데카르트 좌표계가
default)}\label{suv---.-pickup-.---.----cartesnion--default}

ggplot(data = mpg, mapping = aes(x = class, y = hwy)) + geom\_boxplot()
+ coord\_cartesian(xlim = c(0, 5))

\section{xlim은 x축의 범위. 0부터 5까지만 하겠다 근데 좀 애매해짐. x축이
범주형 자료기 때문. 이 경우는 class 왼쪽부터 1,2,3,4,5만 보인듯. 너무
자료가 많아서 보여주고 싶은게 잘 안보일 경우에 이렇게
씀.}\label{xlim-x-.-0-5----.-x---.---class--12345-.----------.}

\section{coord\_fixed로 놓으면 좌표축의 비율을 조절 가능. ratio =
1/2이면 세로가 가로에 비해 2배
김.}\label{coord_fixed-----.-ratio-12----2-.}

ggplot(data = mpg, mapping = aes(x = class, y = hwy)) + geom\_boxplot()
+ coord\_fixed(ratio = 1/1.618)

\section{coord\_flip은 xy를 뒤집음}\label{coord_flip-xy-}

ggplot(data = mpg, mapping = aes(x = class, y = hwy)) + geom\_boxplot()
+ coord\_flip()

\section{지도를 그리는 library(maps)}\label{--librarymaps}

install.packages(``maps'') library(maps) nz \textless{}-
map\_data(``nz'')

ggplot(nz, aes(long, lat, group = group)) + geom\_polygon(fill =
``white'', colour = ``black'') nz summary(nz)

\section{지도는 polygon. 즉 다각형 형식의 자료형임. x은 long(경도) y가
위도.}\label{-polygon.----.-x-long-y-.}

\section{보통 지도는 scale이 가로가 길다. coord\_qucikmap만 치면
그럴듯한 모양이 된다.}\label{--scale--.-coord_qucikmap----.}

ggplot(nz, aes(long, lat, group = group)) + geom\_polygon(fill =
``white'', colour = ``black'') + coord\_quickmap() \# 장난 치기
ggplot(nz, aes(long, lat, group = group)) + geom\_polygon(fill =
``white'', colour = ``black'') + coord\_quickmap() + coord\_flip()

\section{coord\_polar =\textgreater{} 극좌표계
이용.}\label{coord_polar--.}

r\_bar \textless{}- ggplot(data = diamonds) + geom\_bar( mapping = aes(x
= cut, fill = cut), show.legend = FALSE, width = 1 ) +
theme(aspect.ratio = 1) + labs(x = NULL, y = NULL) r\_bar

r\_bar + coord\_polar()

\section{요렇게 하면 극좌표도 그릴 수 있다. 원점으로부터의 거리로서
비율을 나타냄.}\label{-----.----.}

\section{종합 =\textgreater{}}

\section{\texorpdfstring{ggplot(data = ) + ( mapping = aes(), stat = ,
position = ) + +
}{ggplot(data = ) + ( mapping = aes(), stat = , position =  ) +  + }}\label{ggplotdata-mapping-aes-stat-position}

\subsection{Chap 28. Graphics.}\label{chap-28.-graphics.}

\section{For commmunications. 그림이 무엇을 담고 있는지를 덧붙일 필요가
있다.}\label{for-commmunications.-------.}

\section{Title}\label{title}

ggplot(mpg, aes(x = displ, y = hwy)) + geom\_point(aes(color = class)) +
geom\_smooth(se = FALSE) + labs(title = ``Fuel efficiency generally
decreases with engine size'')

\section{labs는 labels의 약자. 여기서 title은
제목.}\label{labs-labels-.--title-.}

\section{Subtitle and caption}\label{subtitle-and-caption}

ggplot(mpg, aes(displ, hwy)) + geom\_point(aes(color = class)) +
geom\_smooth(se = FALSE) + labs( title = ``Fuel efficiency generally
decreases with engine size'', subtitle = ``Two seaters (sports cars) are
an exception because of their light weight'', \# 부제목 caption = ``Data
from fueleconomy.gov'' \# 출처 표기 )

\section{참고로 너무 제목이나 부제가 길면 끝에서 짤린당.}\label{------.}

\section{각 좌표축에도 달수 있음.}\label{---.}

ggplot(mpg, aes(displ, hwy)) + geom\_point(aes(colour = class)) +
geom\_smooth(se = FALSE) + labs( x = ``Engine displacement (L)'', y =
``Highway fuel economy (mpg)'' )

\section{거의 필수적으로 해야할 것은 축에 반드시 단위를 붙여주어야 함.
꼬꼮ㄲ곡}\label{--------.-}

\section{범주형 자료는 예외.}\label{--.}

\section{수식도 집어넣을 수 있음 quote을 사용하면
됨.}\label{----quote--.}

df \textless{}- tibble(x = runif(10), y = runif(10)) ggplot(df, aes(x,
y)) + geom\_point() + labs( x = quote(sum(x{[}i{]} \^{} 2, i == 1, n)),
y = quote(alpha + beta + frac(delta, theta)) \#frac은 분수를 의미. )

\section{더 자세한 사항은 ?poltmath를 보시면
됩니다.}\label{---poltmath--.}

?plotmath

\section{Annotations. 이건 우리가 뒤쪽에서 다룰꺼니까 생략하도록
하겠습니당.}\label{annotations.------.}

\section{포인트에 대해서 설명을 해야할 경우 geom\_text를
사용.}\label{-----geom_text-.}

\section{이건 잠깐 변수를 위한거 공부 ㄴㄴ}\label{-----}

best\_in\_class \textless{}- mpg \%\textgreater{}\% group\_by(class)
\%\textgreater{}\% filter(row\_number(desc(hwy)) == 1) best\_in\_class

ggplot(mpg, aes(x = displ, y = hwy)) + geom\_point(aes(colour = class))
+ geom\_text(aes(label = model), data = best\_in\_class)

\section{ggrepel은 좀더 편함.}\label{ggrepel--.}

install.packages(``ggrepel'') library(ggrepel)

ggplot(mpg, aes(displ, hwy)) + geom\_point(aes(colour = class)) +
geom\_point(size = 3, shape = 1, data = best\_in\_class) +
ggrepel::geom\_label\_repel(aes(label = model), data = best\_in\_class)

\section{Scale.}\label{scale.}

ggplot(mpg, aes(displ, hwy)) + geom\_point(aes(colour = class)) +
scale\_x\_continuous() + scale\_y\_continuous() +
scale\_colour\_discrete()

\section{이게 기본 옵션인데}\label{--}

ggplot(mpg, aes(displ, hwy)) + geom\_point() +
scale\_y\_continuous(breaks = seq(15, 40, by = 5)) \# 15부터 40까지
5마다 틱을 넣어라

ggplot(mpg, aes(displ, hwy)) + geom\_point() +
scale\_x\_continuous(labels = NULL) + scale\_y\_continuous(labels =
NULL)

\section{축 없애기}\label{-}

\section{범위가 큰 자료가 나오는 경우가 있음. 그경우에는 로그로
transform바꾸는게 도움이 될때가 있음.}\label{-----.---transform---.}

ggplot(mpg, aes(x = displ, y = hwy)) + geom\_point() + scale\_y\_log10()

\section{뒤집기}

ggplot(mpg, aes(x = displ, y = hwy)) + geom\_point() +
scale\_x\_reverse()

\section{legend는 기본적으로는 오른족에 붙임. 바꾸고 싶으면 theme이라는
오브젝트 사용. legend.postion 하고 left right, top bottom 등등 으로
붙이기
가능}\label{legend---.---theme--.-legend.postion--left-right-top-bottom----}

ggplot(mpg, aes(displ, hwy)) + geom\_point(aes(colour = class)) +
theme(legend.position = ``left'') \# 만약 더 관심이 있으면 링크에
들어가보자

\subsection{Zooming}\label{zooming}

\section{coord\_cartension을 활용하면 됨.}\label{coord_cartension--.}

ggplot(mpg, mapping = aes(displ, hwy)) + geom\_point(aes(color = class))
+ geom\_smooth() + coord\_cartesian(xlim = c(5, 7), ylim = c(10, 30))

\section{만약에 coord\_cartension이 빠지면, 아예 없는 자료를 smoothing에
넣지않음}\label{-coord_cartension-----smoothing-}

ggplot(mpg, mapping = aes(displ, hwy)) + geom\_point(aes(color = class))
+ geom\_smooth() + xlim(5, 7) + ylim(10, 30)

\section{방금 위코드 실행하면 warning 뜨면서 자료가 빠졌다고
알려줌.}\label{---warning----.}

\section{뒷부분은 생략하고 바로 Themes으로 넘어감.}\label{---themes-.}

\subsection{Theme}\label{theme}

\section{아무것도 안하면 배경은 회색, 선은 파란색\ldots{} 등등. 근데
Theme은 이런거 바꿀 수 있음.}\label{------.--theme----.}

ggplot(mpg, aes(displ, hwy)) + geom\_point(aes(color = class)) +
geom\_smooth(se = FALSE) + theme\_bw() \# bw는 blakck \& white. \#
어떤게 있는지 궁금하면 교재를 참조하면 됨. \# 이런것을 참조하면 깔끔한
보고서를 쓸수 있음. 특히 회색 배경은 프린터할때 x되는 경우가 많음
그럴때는 bw 테마를 쓰는게 나음.

\section{그림만 따로 저장하고 싶을때.}\label{---.}

\section{만든 다음에 바로 ggsave하면 됨. 확장자도 png. pdf 등등으로 바꿀
수 있음}\label{---ggsave-.--png.-pdf----}

ggplot(mpg, aes(displ, hwy)) + geom\_point() ggsave(``my\_plot.pdf'')

\section{만약에 그림을 변수에 저장해 놓았다면 나중에 저장해도 됨. 아니면
바로 저장해줘야뎀.}\label{-------.---.}

\section{좀 지나고 저장하려면 plot = 로 따로 저장해주어야
함.}\label{---plot----.}

?ggsave

\section{좀더 편한건 Data visulaization with ggplot :: CHAT SHEET를 보면
됨.}\label{--data-visulaization-with-ggplot-chat-sheet--.}


\end{document}
